\section{Implementation}
This section will outline the employed software and hardware resources of the system, explain the data preprocessing, and describe the model architecture of our neural networks in detail.

\subsection{Software and Hardware Resources}

	The language identification system is implemented in Python. We used the open source deep learning framework Keras\cite{chollet2015keras} with the TensorFlow\cite{abadi2016tensorflow} backend for training our neural networks. Keras provides us with a set of higher level machine learning primitives such as convolutional layers and optimizing algorithms without abstracting too much away. Internally it builds on Google's open source numerical operation library TensorFlow which is optimized to quickly compute large multidimensional data on GPUs.
		All trainings and experiments were executed on two cuda-enabled GPU machines belonging to the Internet Technologies and Systems chair. Details can be found in table \ref{tab:hardware}.
		
	\begin{table}[h]
	\centering
	\begin{tabularx}{\textwidth}{lll}
	\toprule
	  		& Machine A 					& Machine B \\ \midrule
	OS  	& Ubuntu Linux 14.04 		& Ubuntu Linux 16.04 \\
	CPU  	& Intel\textsuperscript{\textregistered} Core\textsuperscript{\texttrademark} i7-4790K @4GHz 	& AMD FX\textsuperscript{\texttrademark}-8370  @ 4GHz \\
	RAM  	& 16GB 						& 32GB \\
	GPU  	& Nvidia GeForc\textsuperscript{\textregistered}e GTX 980 	& Nvidia Titan X \\
	VRM  	& 4GB 						& 12GB \\
	\bottomrule
	\end{tabularx}
	\caption{Hardware ressources used in training the neural network.}
	\label{tab:hardware}
	\end{table}
	 
	We designed the system in three parts. 
	\begin{enumerate}
		\item The preprocessor. Its job is to download, extract, and convert the data. The preprocessor clips the audio tracks for raw video footage and converts them into WAV files before ultimately converting the data into PNG images for the training. Details can be found in section \ref{sec:data_processing}.
		\item The trainer. 
		\item The evaluator.
	\end{enumerate}

\subsection{Data Preprocessing}
\label{sec:data_processing}

	

    \begin{itemize}
        \item Spectrogram Generation
        \item Data Segmentation
        \item (Augmentation)
        \item train / test split
		\item number of training samples
    \end{itemize}

\subsection{CNN Architecture}

    \begin{itemize}
        \item Layer Table
        \item Transfer Learning / Fine-tuning
        \item Variations
    \end{itemize}

\subsection{CRNN Architecture}

    \begin{itemize}
        \item Layer Table
        \item Architecture Image
        \item Conv Features to time steps
        \item Transfer Learning / Fine-tuning
        \item 
    \end{itemize}
