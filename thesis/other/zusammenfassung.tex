\section*{\LARGE Zusammenfassung}
Mit zunehmender Verf\"ugbarkeit von Spracheingabesystemen in Computern, sind immer mehr Nutzer auf verl\"assliche Spracherkennungsalgorithmen angewiesen. Dabei ist automatische Sprachidentifizierung der erste und wichtigste Schritt f\"ur die Spracherkennung. Ohne automatische Sprachidentifizierung sind alle weiteren Erkennungsschritt nutzlos, da weder Sprachkl\"ange richtige verarbeitet noch Grammatikregeln ordentlich angewandt werden.

Ein zweiter aktueller Trend in der Informatik ist die erfolgreiche Anwendung von tiefen neuronalen Netzwerken f\"ur eine Vielzahl von Aufgabenstellungen. In dieser Arbeit stellen wir ein hybrides, neuronales Netzwerksystem zur automatischen Sprachidentifizierung von Sprachaufnahmen mithilfe von \emph{deep learning} Techniken vor. Im speziellen wenden wir mehrere \emph{convolutional recurrent neural network} Modelle auf menschliche Spracheeingabe an und evaluieren deren Robustheit in verschiedenen Ger\"auschumgebungen.

\emph{Convolutional neural networks} haben innerhalb der computerbasierten Bilderkennungsforschung gro{\ss}e Durchbr\"uche erzielt. Deshalb basieren wir unsere Forschung auf deren etablierte Modelarchitekturen, wie das Inception Netzwerk\cite{szegedy2015going} and \"uberf\"uhren unsere audiobasierte Forschungsfrage in die Bildverarbeitungsdom\"ane. Dabei untersuchen wir die Effektivit\"at von Spektrogrammbildern als n\"utzliches Eingabemedium. Wir stellen weitere Audiorepr\"asentationen und weiterf\"uhrende Publikationen vor.

\emph{Deep learning} Systeme kommt besonders die Verf\"ugbarkeit von gro{\ss}en Datens\"atzen zu gute. Wir trainieren unsere Modelle auf mehr als \num{1000} Stunden an Sprachaufnahmen in sechs verschiedenen Sprachen: Englisch, Deutsch, Franz\"osisch, Spanisch, Mandarin und Russisch. Wir sammeln und verarbeiten Daten aus Reden und Sitzungen des Europ\"aischen Parlaments, sowie von Nachrichtensender, wie beispielsweise der BBC, die auf YouTube zur Verf\"ugung gestellt werden.

Unser leistungsf\"ahigstes \emph{convolutional recurrent neural network} Modell erreicht eine Erkennungsgenauigkeit und ein F-Ma{\ss} von \SI{96}{\percent} auf dem Nachrichtendatensatz. Unser Ansatz stellt eine konstante Verbesserungen gegen\"uber allen Nullmessungen auf Grundlage von \emph{convolutional neural networks} dar. Wir evaluieren unsere Modellen in verschieden St\"orger\"auchszenarien mit k\"unstlich ver\"anderten Daten. Dabei f\"ugen wir Rauschen, Geknister und Hintergrundmusik zu unseren Daten hinzu und messen einen Genauigkeitsverlust von jeweils 5 Prozentpunkten (PP), 3 PP und 7 PP gegen\"uber den Originaldaten. Auf dem kleineren EU-Datensatz erreichen wir eine Genauigkeit und ein F-Ma{\ss} von \SI{98}{\percent}.